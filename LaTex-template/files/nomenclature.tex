% you can create hyperlinks between acronyms in the bulk text and their descriptions, which are placed inside this nomenclature-chapter

% the "*" after "chapter" is used to suppress the chapter number
\chapter*{Nomenclature}   

% include Nomenclature into the table of contents
\addcontentsline{toc}{chapter}{Nomenclature}

% add label to reference to nomenclature
\label{text:nomenclature}

% The nomenclature can be divided into different categories, e.g. via sections
\section*{Abbreviations}

\begin{acronym}[$leeeeeeength_of_tabulator$]    % number of "eee"s defines the length of tabulator between acronym and its explanation
\setlength{\itemsep}{0mm}

% now the acronyms can be defined, syntax:
% \acro{input for \ac{..} }[displayed form]{\acroextra{\hyperref[label at first occurrence]{long form, shown only here}}}

% sample data
\acro{tg}[TG]{\acroextra{\hyperref[def:tg]{Thermogravimetry}}}


\end{acronym}

% to place the acronym inside the text, you can write one of the following:
% \ac{..}   % initially full, then short form
% \acf{..}  % full form
% \acs{..}  % short form


% Normally, any \label outside an object (like figures or tables) points to the current section title.
% To have a link point to the exact line where the \label is defined (e.g. to point exactly to the first usage of a new acronym), you can use the command \phantomsection just before the label and acronym
%% bla bla
%% \phantomsection \label{def:tg} \acs{tg}
%% bla bla


\section*{Greek Symbols}
\begin{acronym}[$leeeeeeength_of_tabulator$]
\setlength{\itemsep}{0mm}

%% start listing acronyms
\acro{eps}[$\epsilon$]{\acroextra{\hyperref[def:eps]{Die Permittivität $\epsilon$, auch bekannt als dielektrische Konstante, ist ein Mass für die Durchlässigkeit eines Materials für elektrische Felder.}}}

% to check if symbols not used in the text are omitted from the nomenclature, an additional entry is placed here:
\acro{om}[$\omega$]{\acroextra{\hyperref[def:om]{Elektrischer Widerstand in [Ohm].}}}

\end{acronym} 